\documentclass[a4paper, 11pt]{article}

\usepackage[justification=centering]{caption}
\usepackage[catalan]{babel}
\usepackage[utf8]{inputenc}
\usepackage{mathtools}
\usepackage{microtype}
\usepackage{pythontex}
\usepackage{amsfonts}
\usepackage{enumitem}
\usepackage{fancyhdr}
\usepackage{geometry}
\usepackage{hyperref}
\usepackage{graphicx}
\usepackage{listings}
\usepackage{verbatim}
\usepackage{amsmath}
\usepackage{float}
\usepackage[table,xcdraw]{xcolor}

\definecolor{codegreen}{rgb}{0,0.6,0}
\definecolor{codegray}{rgb}{0.5,0.5,0.5}
\definecolor{codepurple}{rgb}{0.58,0,0.82}
\definecolor{backcolour}{rgb}{0.95,0.95,0.92}

\lstdefinestyle{mystyle}{
    backgroundcolor=\color{backcolour},
    commentstyle=\color{codegreen},
    keywordstyle=\color{magenta},
    numberstyle=\tiny\color{codegray},
    stringstyle=\color{codepurple},
    basicstyle=\ttfamily\footnotesize,
    breakatwhitespace=false,
    breaklines=true,
    captionpos=b,
    keepspaces=true,
    numbers=left,
    numbersep=5pt,
    showspaces=false,
    showstringspaces=false,
    showtabs=false,
    tabsize=2
}

\lstset{style=mystyle}

\DisableLigatures{encoding = *, family = *}

\geometry{left=25mm, right=25mm, top=25mm, bottom=25mm}
\pagestyle{fancy}
\fancyhf{}
\lhead{Joel, Jordi, i Oriol}
\rhead{Aprenentatge Computacional}
\cfoot{\thepage}

\renewcommand{\headrulewidth}{0.6pt}
\renewcommand{\footrulewidth}{0.6pt}
\renewcommand{\baselinestretch}{1.5}
\setlength{\headheight}{13.6pt}

\title{\Huge{\textbf{Pràctica 2: Etiquetatge de roba}}}
\author{\Large{Joel Guevara Lopez, 1564581}
        \\\Large{Jordi Morales Casas, 1564921}
        \\\Large{Oriol Benítez Bravo, 1566931}}
\date{Aprenentatge Computacional, III MatCAD\\ \vspace{6pt} Octubre 2021}


\begin{document}

    \maketitle

    \section{Descripció de la base de dades}

    La nostra base de dades consisteix en un recull de valors que ens dona tota la informació sobre
    diferents característiques d'un cotxe utilitzant dos tipus de combustible.

    Les característiques que ens donen, s'agrupen en distància, consum, velocitat, temperatura dins
    del cotxe, temperatura fora del cotxe, característiques especials, tipus de combustible, si té
    activat o no l'aire condicionat, si plou o no, si fa Sol o no, si ha fet provisió de gasolina
    durant el trajecte i quin tipus de gas s'ha provisió de.

    En el primer anàlisi de les dades, han sigut eliminades les columnes de característiques
    especials, ja que no aporta informació (només repeteix la informació de la pluja i del Sol) i el
     tipus de gas fet provisió de, ja que és el mateix que en la columna de tipus de combustible.

    Després d'aquest primer anàlisis, s'ha conclòs a utilitzar com a variable a predir el consum que
     el cotxe ha realitzat en un trajecte, pel fet que és l'única variable que ens aporta una
     informació rellevant sobre el vehicle i que ens pot ajudar a decidir sobre el que ens ajuda o
     no a tenir menys despeses.

    Com que el consum no té relació amb si s'ha o no fet provisió de gasolina, ja que no coneixem
    quants litres hi havia inicialment, s'ha eliminat de l'anàlisi.

    S'ha considerat crear la variable temps a partir dels valors de distància i velocitat, ja que
    probablement té alguna relació amb el consum.

    Un cop preparada la base de dades, s'han emplenat els valors buits amb la mitjana dels valors
    que hi ha en aquella columna i s'han passat a binari els valors referents al tipus de gas.
    D'aquesta manera, totes les dades tenen valors numèrics.

    \subsection{Correlació de les variables}

    Inicialment, les variables tenen una correlació molt baixa amb la variable consum (figura \ref{fig:C_I}).


    \begin{figure}[H]
        \centering
        \includegraphics[scale=1]{correlacions_inicials.png}
        \caption{Correlacions inicials de les variables de la base de dades.}
        \label{fig:C_I}
    \end{figure}

    Per poder predir, necessitem una correlació més gran, per aconseguir-ho serà necessari aplicar
    transformacions a les dades per poder aconseguir trobar valors més correlacionals.

    Primerament observem que els histogrames de les correlacions es poden veure característiques en
    les quals els valors estan molt concentrats en molt poc espai, per tant una transformació útil
    pot ser aplicar un 1/valor en aquests casos (figura \ref{fig:H_I}). Es pot observar que les
    variables distancia i velocitat haurien de comportar-se millor amb la transformació prèviament
    esmentada.

    \begin{figure}[H]
        \centering
        \includegraphics[scale=0.25]{histogrames_inicials.png}
        \caption{Histogrames dels valors inicials de les variables de la base de dades.}
        \label{fig:H_I}
    \end{figure}

    Es por observar en les gràfiques de distribució de les variables que nomes podem assegurar que
    segueixin una distribució gaussiana les variables consum, velocitat, temperatura a dins i
    temperatura a fora.

    Podem observar en les taules de correlacions amb la transformació (figura \ref{fig:C_T}) que la
    millora és gran, i per tant podem aprofitar aquesta millora per poder decidir quines
    característiques ens ajudaran a poder generar la regressió.

    \begin{figure}[H]
        \centering
        \includegraphics[scale=1]{correlacions_transformades.png}
        \caption{Correlacions amb les transformacions de les variables de la base de dades.}
        \label{fig:C_T}
    \end{figure}

    És observable que les variables que sembla que poden ser més útils per poder fer una regressió
    seran: la distancia, la velocitat, la pluja i el temps.


    Podem observar en els histogrames (figura \ref{fig:H_T}) que la distribució en aquestes variables,
    ha sigut millorada i s'han dispersat de forma uniforme els valors, facilitant la correlació.

    \begin{figure}[H]
        \centering
        \includegraphics[scale=0.25]{histogrames_transformats.png}
        \caption{Histogrames dels valors transformats de les variables de la base de dades.}
        \label{fig:H_T}
    \end{figure}





\newpage
    \section{Avaluació de la importància dels diferents atributs}


        Per començar a fer regressió sobre les dades cal decidir amb quines característiques ens
        quedem i quines podem descartar. Aquest pas és essencial perquè les conclusions finals
        dependran de les decisions que prenem en aquest punt de l'anàlisi.

        Un primer mètode per avaluar com són de rellevants els diferents atributs pot ser fer una
        regressió lineal molt simple per cadascun i calcular l'error quadràtic mig (\textit{MSE})
        que donen els diferents models.
        Recordem que aquest error és un estimador de la diferència mitjana al quadrat entre la
        predicció del model i el valor que realment té, i es calcula seguint aquesta fórmula:
        \begin{gather*}
          MSE = \frac{1}{n}\sum_{i=1}^n (\hat{Y}_i - Y_i)^2
        \end{gather*}


        Si provem això amb el nostre \textit{dataset} obtenim els resultats següents:\\

        \begin{table}[H]
            \centering
            \begin{tabular}{lllllllll}
                \hline
                \multicolumn{1}{|l|}{\textbf{Dist.}} & \multicolumn{1}{l|}{\textbf{Vel.}} & \multicolumn{1}{l|}{\textbf{Tª int.}} & \multicolumn{1}{l|}{\textbf{Tª ext.}} & \multicolumn{1}{l|}{\textbf{Tipus combust.}} & \multicolumn{1}{l|}{\textbf{AC}} & \multicolumn{1}{l|}{\textbf{Pluja}} & \multicolumn{1}{l|}{\textbf{Sol}} & \multicolumn{1}{l|}{\textbf{Temps}} \\ \hline
                0.5405                                   & 0.5783                                  & 0.6388                                & 0.5471                                & 0.6050                                   & 0.6002                           & 0.5890                              & 0.6243                            & 0.5209
            \end{tabular}
        \end{table}


        L'atribut amb menor MSE, $0.5209$, és el temps. Això és interessant, ja que aquest atribut
        es deriva de la distància i la velocitat, que tenen $0.5405$ i $0.5783$ respectivament. La
        temperatura exterior també a donat un dels errors més baixos mentre que la interior sembla
        ser l’atribut menys útil, però en general tots els errors són força semblants. Cal
        preguntar-se si potser descartar atributs sense tenir en compte com poden estar relacionats
        entre si és bona idea pel nostre cas concret.\\


        Per tal de trobar quins són els atributs més importants hem fet servir un mètode de backward
        selection: Començant amb el model sencer, fem un seguit de regressions eliminant cada cop
        l'atribut que menys informació aporti. Per prendre aquesta decisió ens fixem en el p-value
        associat a cadascun. Un p-value molt gran ens indica que no és estadísticament significant
        pel model i per tant podem prescindir d'aquest atribut. Deixem d’eliminar variables quan
        tots els p-values siguin prou baixos (normalment menor que $0.05$).\\

        La taula següent mostra aquest procés d'eliminació, fila per fila. Cal dir que els valors
        que apareixem com 0 en realitat si tenen un valor, però surt de la precisió que mostra la
        llibreria.
        En qualsevol cas no afecta al resultat final donat que busquem precisament els valors més
        grans.\\


        \begin{table}[H]
            \centering
            \begin{tabular}{|l|l|l|l|l|l|l|l|l|}
                \hline
                \textbf{Dist.} & \textbf{Vel.}            & \textbf{Tª int.} & \textbf{Tª ext.} & \textbf{Tipus combust.}  & \textbf{AC} & \textbf{Pluja} & \textbf{Sol}                        & \textbf{Temps} \\ \hline
                0.000              & 0.966                    & 0.357                               & 0.000                              & 0.314                    & 0.128                         & 0.001          & 0.520                    & 0.079          \\ \hline
                0.000              & \cellcolor[HTML]{C0C0C0} & 0.354                               & 0.000                              & 0.313                    & 0.125                         & 0.001          & 0.518                    & 0.012          \\ \hline
                0.000              & \cellcolor[HTML]{C0C0C0} & 0.316                               & 0.000                              & 0.316                    & 0.125                         & 0.001          & \cellcolor[HTML]{C0C0C0} & 0.011          \\ \hline
                0.000              & \cellcolor[HTML]{C0C0C0} & 0.353                               & 0.000                              & \cellcolor[HTML]{C0C0C0} & 0.147                         & 0.001          & \cellcolor[HTML]{C0C0C0} & 0.009          \\ \hline
                0.000              & \cellcolor[HTML]{C0C0C0} & \cellcolor[HTML]{C0C0C0}            & 0.000                              & \cellcolor[HTML]{C0C0C0} & 0.208                         & 0.001          & \cellcolor[HTML]{C0C0C0} & 0.009          \\ \hline
                0.000              & \cellcolor[HTML]{C0C0C0} & \cellcolor[HTML]{C0C0C0}            & 0.000                              & \cellcolor[HTML]{C0C0C0} & \cellcolor[HTML]{C0C0C0}      & 0.000          & \cellcolor[HTML]{C0C0C0} & 0.007          \\ \hline
            \end{tabular}
        \end{table}

        Després de repetir aquest procés 6 vegades, obtenim que els atributs més rellevants són la
        distància (\textit{distance}), la temperatura a l'exterior (\textit{temp\_outside}), si
        plou o no (\textit{rain}) i el temps del recorregut (\textit{time}).\\


        Amb aquesta configuració, si fem un model lineal que segueixi la fórmula
        \begin{gather*}
          Consume \sim Distance + Temp\_Outside + Rain + Time
        \end{gather*}
        es pot observar en la sortida de la execució (imatge \ref{fig:R_L}) els p-values associats
        i a sota l'R2 Score de 0.669054, que és una mesura directa de la precisió ($66.9054\%$).

        \begin{figure}[H]
            \centering
            \includegraphics[scale=0.75]{SalidaModeloLineal.png}
            \caption{Resultats de la regressió lineal.}
            \label{fig:R_L}
        \end{figure}

        Per tal de intentar millorar aquest resultat pasem a un model polinomial, de grau 2, amb la
        fórmula
        \begin{gather*}
          Consume \sim Distance + Temp\_Outside + Temp\_Outside^2 + Rain + Rain^2 + Time + Time^2
        \end{gather*}

        Hem decidit no incluir l'atribut \textit{Distance} elevat al quadrat degut a que dona pitjor
        resultat que si no s'inclou (Tot i que continuaria sent millor que el model lineal). Ara,
        com es pot veure a la sortida del programa (imatge \ref{fig:R_L}), la precisió puja fins al
        $71.2489\%$.

        \begin{figure}[H]
            \centering
            \includegraphics[scale=0.75]{SalidaModeloPolinomial.png}
            \caption{Resultats de la regressió polinomial.}
            \label{fig:R_L}
        \end{figure}



        \paragraph{4. Com influeix la normalització en la regressió?}
        Es pot aplicar la normalització sobre el model generat, però no obtenim millores en el
        resultat. Per normalitzar s'aplica la següent equació:

        \begin{equation}
            \frac{x-\overline{x}}{\sigma}
        \end{equation}
        on \texttt{x} són tots els valors del \textit{dataset}. Aquesta formula centralitza les
        dades a la mitja.


        \paragraph{6. Si s'aplica un PCA, a quants components es redueix l'espai? Per què?}

    \section{Descens del gradient}
        \paragraph{Com influeixen tots els paràmetres en el procés de descens? Quins valors de
        learning rate convergeixen més ràpid a la solució òptima? Com influeix la inicialització
        del model en el resultat final?}


        \paragraph{Quines funcions polinomials (de diferent grau, de diferents combinacions
        d'atributs, ...) heu escollit per ser apreses amb el vostre descens del gradient? quina ha
        donat el millor resultat (en error i rapidesa en convergència)?}

        \paragraph{Utilitzeu el regularitzador en la fòrmula de funció de cost i descens del
        gradient i proveu polinomis de diferent grau. Com afecta el valor del regularitzador?}

        \paragraph{Quina diferència (quantitativa i qualitativa) hi ha entre el vostre regressor
        i el de la llibreria?}

        \paragraph{Té sentit el model (polinomial) trobat quan es visualitza sobre les dades?}

        \paragraph{Ajuda la visualització a identificar aquelles mostres per a les que el regressor
        obté els pitjors resultats de predicció?}

\end{document}
